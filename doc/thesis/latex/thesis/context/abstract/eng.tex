% ------------------------------------------------
\StartAbstract
% ------------------------------------------------
In dialogue generation, the naturalness of responses is key for effective human-machine interaction, significantly enhancing user experience. Personalized dialogue generation poses even greater challenges, as the responses must be coherent and consistent with the user's personal traits or persona descriptions. In this study, we propose a novel method named \textbf{MUDI} (\textbf{Mu}ltiple \textbf{Di}scourse Relations Graph Learning) aimed at effectively modeling and integrating discourse relations and persona information within the context of personalized dialogue generation. We initially utilize Large Language Models to assist in annotating discourse relations and to transform dialogue data into structured dialogue graphs. We employ our newly proposed DialogueGAT as the graph encoder, which captures implicit discourse relations within this structure. Persona descriptions are also encoded into fully-connected persona graphs, facilitating the capture of semantic relationships between persona elements. An Attention-Based Feature Fusion method integrates data from both graphs, creating a personalized, coherence-aware dialogue representation. In the personalized response generation phase, a Prompt-based mechanism and a Coherence-Aware Attention strategy are implemented to enhance the decoder's consideration of discourse relations. Our experiments and case studies demonstrate significant improvements in the quality of personalized responses, making them more coherent, aligned with persona, and natural, thus resembling human-like dialogue exchanges.

% In dialogue generation, high naturalness of responses is crucial for effective human-machine interaction, significantly enhancing user experience. Personalized response generation poses even greater challenges, as the responses must be coherent and consistent with the user's personal traits or persona descriptions. In this study, we propose a novel method named \textbf{MUDI} (\textbf{Mu}ltiple \textbf{Di}scourse Relations Graph Learning) aimed at effectively modeling and integrating discourse relations and persona information within the context of personalized dialogue generation. Initially, to enable the model to learn discourse coherence relations within dialogues, we utilize Large Language Models to assist in annotating discourse relations in the dialogue data. Inspired by prior research, we transform the dialogue data into a dialogue graph, then apply our proposed DialogueGAT as an encoder to capture the implicit discourse coherence relations within the dialogue structure. Additionally, we transform persona descriptions into fully-connected persona graphs through the graph encoder to capture semantic relationships between persona sentences. Finally, we employ an attention mechanism-based feature fusion method to integrate information from both the dialogue and persona graphs, creating a personalized, coherence-aware dialogue representation. We also capture the implicit relationships between dialogue context and persona descriptions at the semantic and paragraph levels through a text encoder. In the response generation phase, we first employ a prompt-based conditional dialogue generation mechanism, using carefully designed prompts to guide the generation of personalized responses. We introduce a coherence-enhanced attention mechanism, combining learnable embeddings and text representations to enhance the decoder's ability to consider discourse relations when predicting the next word, allowing it to consider vocabulary that carries implicit coherence information during text generation. Experimental results and case studies demonstrate that our method significantly improves the quality of personalized dialogue responses, effectively integrating discourse relations and persona information, making the responses more coherent, adhering to persona settings, and more natural, resembling human-like dialogue responses.

% ------------------------------------------------
\EndAbstract
% ------------------------------------------------
